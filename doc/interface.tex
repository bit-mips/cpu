\documentclass{article}

\usepackage[a4paper,margin=1in]{geometry}
\usepackage[UTF8]{ctex}
\usepackage{tikz}
\usepackage{tikz-timing}

\newenvironment{signals}{
	\begin{center}
		\begin{tabular}{| c | c | c | c |}
			\hline
			信号 & 方向 & 宽度 & 描述 \\ \hline
}{
		\end{tabular}
	\end{center}
}

\newcommand\sigin{Input}
\newcommand\sigout{Output}

\begin{document}

\section{IF}

\subsection{时钟复位信号}

\begin{signals}
	clock & \sigin & 1 & 时钟 \\ \hline
	reset & \sigin & 1 & 同步复位,高电平有效 \\ \hline
\end{signals}

\subsection{指令存储接口}

\subsubsection{信号}

\begin{signals}
	ibus\_address & \sigout & 32 & 地址 \\ \hline
	ibus\_read & \sigout & 1 & 读使能 \\ \hline
	ibus\_data & \sigin & 32 & 读数据 \\ \hline
	ibus\_stall & \sigin & 1 & 暂停 \\ \hline
	ibus\_error & \sigin & 1 & 错误 \\ \hline
\end{signals}

\subsubsection{时序}

IF模块在第一个上升沿置ibus\_read为高,同时给出ibus\_address,并维持至下一个上升沿。
存储器在第二个上升沿给出数据,若未就绪则置ibus\_stall为高。
IF模块第三个上升沿读ibus\_data。

参考图\ref{tt:ifmem}。

\begin{figure}[h]
	\centering
	\begin{tikztimingtable}
		clock &         HCCC    CC    CC    CC    CC    CC    CC    CC    CC    CC    CC \\
		ibus\_read &    LLHH    LL    HH    HH    LL    HH    HH    LL    HH    LL    LL \\
		ibus\_address & UUDD{A0}UU    DD{A1}DD{A2}UU    DD{A3}DD{A4}UU    DD{A5}UU    UU \\
		ibus\_stall &   LLLL    LL    LL    LL    LL    LL    HH    LL    LL    LL    LL \\
		ibus\_data &    UUUU    DD{D0}UU    DD{D1}DD{D2}UU    UU    DD{D3}DD{D4}DD{D5}UU \\
	\end{tikztimingtable}
	\caption{IF指令存储接口时序}
	\label{tt:ifmem}
\end{figure}

\subsection{下级FIFO接口}

\subsubsection{信号}

\begin{signals}
	output\_address & \sigout & 32 & 指令地址 \\ \hline
	output\_instruction & \sigout & 32 & 指令数据 \\ \hline
	output\_valid & \sigout & 32 & 指令数据有效 \\ \hline
	output\_full & \sigin & 32 & FIFO满反馈 \\ \hline
\end{signals}

\subsubsection{时序}

IF模块需要向FIFO写入时,将数据送至output\_address及output\_instruction并使能output\_valid信号。
检测到output\_full信号时需要停止写入,并等待full信号撤消后继续。

参考图\ref{tt:iffifo}。

\begin{figure}[h]
	\centering
	\begin{tikztimingtable}
		clock &               HCCC    CCCC    CC    CC    CCCCCC    CC \\
		output\_address &     UUDD{A0}UUDD{A1}DD{A2}DD{A3}UUUUDD{I4}UU \\
		output\_instruction & UUDD{I0}UUDD{I1}DD{I2}DD{I3}UUUUDD{I4}UU \\
		output\_valid &       LLHH    LLHH    HH    HH    LLLLHH    LL \\
		output\_full &        LLLL    LLLL    LL    HH    HHLLLL    LL \\
	\end{tikztimingtable}
	\caption{IF FIFO接口时序}
	\label{tt:iffifo}
\end{figure}

\subsection{转移接口}

\begin{signals}
	branch\_address & \sigin & 32 & 跳转地址 \\ \hline
	branch\_valid & \sigin & 32 & 跳转有效 \\ \hline
\end{signals}

\end{document}
