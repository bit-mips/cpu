\documentclass{article}

\usepackage[a4paper,margin=1in]{geometry}
\usepackage[UTF8]{ctex}
\usepackage{tikz}
\usepackage{tikz-timing}

\newenvironment{signals}{
	\begin{center}
		\begin{tabular}{| c | c | c | c |}
			\hline
			信号 & 方向 & 宽度 & 描述 \\ \hline
}{
		\end{tabular}
	\end{center}
}

\newcommand\sigin{Input}
\newcommand\sigout{Output}

\begin{document}

\section{FIFO}

\subsection{时钟复位信号}

\begin{signals}
	clock & \sigin & 1 & 时钟 \\ \hline
	reset & \sigin & 1 & 同步复位,高电平有效 \\ \hline
\end{signals}

\subsection{输入接口}

\subsubsection{信号}

\begin{signals}
	input\_data & \sigin & n & 数据 \\ \hline
	input\_valid & \sigin & 1 & 数据有效 \\ \hline
	input\_full & \sigout & 1 & 满反馈 \\ \hline
\end{signals}

\subsubsection{时序}

参考图\ref{tt:fifoin}。

\begin{figure}[h]
	\centering
	\begin{tikztimingtable}
		clock (\sigin) &        HCCC    CCCC    CC    CCCCCC    CC    CC \\
		input\_data (\sigin) &  UUDD{D0}UUDD{D1}DD{D2}DDDDDD{D3}DD{D4}UU \\
		input\_valid (\sigin) & LLHH    LLHH    HH    HHHHHH    HH    LL \\
		input\_full (\sigout) & LLLL    LLLL    LL    HHHHLL    LL    LL \\
	\end{tikztimingtable}
	\caption{FIFO输入接口时序}
	\label{tt:fifoin}
\end{figure}

\subsection{输出接口}

\subsubsection{信号}

\begin{signals}
	output\_data & \sigout & n & 数据 \\ \hline
	output\_valid & \sigout & 1 & 数据有效 \\ \hline
	output\_full & \sigin & 1 & 满反馈 \\ \hline
\end{signals}

\subsubsection{时序}

参考图\ref{tt:fifoin}。

\begin{figure}[h]
	\centering
	\begin{tikztimingtable}
		clock (\sigin) &          HCCC    CCCC    CC    CCCCCC    CC    CC \\
		output\_data (\sigout) &  UUDD{D0}UUDD{D1}DD{D2}DDDDDD{D3}DD{D4}UU \\
		output\_valid (\sigout) & LLHH    LLHH    HH    HHHHHH    HH    LL \\
		output\_full (\sigin)  &  LLLL    LLLL    LL    HHHHLL    LL    LL \\
	\end{tikztimingtable}
	\caption{FIFO输出接口时序}
	\label{tt:fifoin}
\end{figure}

\section{IF}

\subsection{时钟复位信号}

\begin{signals}
	clock & \sigin & 1 & 时钟 \\ \hline
	reset & \sigin & 1 & 同步复位,高电平有效 \\ \hline
\end{signals}

\subsection{指令存储接口}

\subsubsection{信号}

\begin{signals}
	ibus\_address & \sigout & 32 & 地址 \\ \hline
	ibus\_read & \sigout & 1 & 读使能 \\ \hline
	ibus\_data & \sigin & 32 & 读数据 \\ \hline
	ibus\_stall & \sigin & 1 & 暂停 \\ \hline
	ibus\_error & \sigin & 1 & 错误 \\ \hline
\end{signals}

\subsubsection{时序}

IF模块在第一个上升沿置ibus\_read为高,同时给出ibus\_address,并维持至下一个上升沿。
存储器在第二个上升沿给出数据,若未就绪则置ibus\_stall为高。模块在检测到ibus\_stall信号时保持输出不变。
IF模块第三个上升沿读ibus\_data。

参考图\ref{tt:ifmem}。

\begin{figure}[h]
	\centering
	\begin{tikztimingtable}
		clock &         HCCC    CC    CC    CC    CC    CC    CCCC    CC    CC    CC \\
		ibus\_read &    LLHH    LL    HH    HH    LL    HH    HHHH    HH    LL    LL \\
		ibus\_address & UUDD{A0}UU    DD{A1}DD{A2}UU    DD{A3}DDDD{A4}DD{A5}UU    UU \\
		ibus\_stall &   LLLL    LL    LL    LL    LL    LL    HHLL    LL    LL    LL \\
		ibus\_data &    UUUU    DD{D0}UU    DD{D1}DD{D2}UU    UUDD{D3}DD{D4}DD{D5}UU \\
	\end{tikztimingtable}
	\caption{IF指令存储接口时序}
	\label{tt:ifmem}
\end{figure}

\subsection{下级FIFO接口}

\subsubsection{信号}

\begin{signals}
	output\_address & \sigout & 32 & 指令地址 \\ \hline
	output\_instruction & \sigout & 32 & 指令数据 \\ \hline
	output\_valid & \sigout & 1 & 指令数据有效 \\ \hline
	output\_full & \sigin & 1 & FIFO满反馈 \\ \hline
\end{signals}

\subsubsection{时序}

IF模块需要向FIFO写入时,将数据送至output\_address及output\_instruction并使能output\_valid信号。
检测到output\_full信号时需要停止写入,并等待full信号撤消后继续。

参考图\ref{tt:iffifo}。

\begin{figure}[h]
	\centering
	\begin{tikztimingtable}
		clock &               HCCC    CCCC    CC    CCCCCC    CC    CC \\
		output\_address &     UUDD{A0}UUDD{A1}DD{A2}DDDDDD{A3}DD{A4}UU \\
		output\_instruction & UUDD{I0}UUDD{I1}DD{I2}DDDDDD{I3}DD{I4}UU \\
		output\_valid &       LLHH    LLHH    HH    HHHHHH    HH    LL \\
		output\_full &        LLLL    LLLL    LL    HHHHLL    LL    LL \\
	\end{tikztimingtable}
	\caption{IF FIFO接口时序}
	\label{tt:iffifo}
\end{figure}

\subsection{转移接口}

\begin{signals}
	branch\_address & \sigin & 32 & 跳转目标 \\ \hline
	branch\_inst\_addr & \sigin & 32 & 跳转指令地址 \\ \hline
	branch\_valid & \sigin & 1 & 跳转有效 \\ \hline
\end{signals}

\section{ID}

\subsection{时钟复位信号}

\begin{signals}
	clock & \sigin & 1 & 时钟 \\ \hline
	reset & \sigin & 1 & 同步复位,高电平有效 \\ \hline
\end{signals}

\subsection{Regfile接口}

\subsubsection{信号}

\begin{signals}
	reg\_s & \sigout & 5 & rs \\ \hline
	reg\_t & \sigout & 5 & rt \\ \hline
	reg\_id\_d & \sigout & 5 & 锁寄存器 \\ \hline
	reg\_s\_data & \sigin & 32 & rs数据 \\ \hline
	reg\_t\_data & \sigin & 32 & rt数据 \\ \hline
	reg\_stall & \sigin & 1 & 暂停信号 \\ \hline
\end{signals}

\subsubsection{时序}

参考图\ref{tt:idreg}。

\begin{figure}[h]
	\centering
	\begin{tikztimingtable}
		clock &        HCCC    CC     CC     CCCC     CC     CC     CC \\
		reg\_s &       UUDD{S0}DD{S1} DD{S2} DDDD{S3} DD{S4} UU     UU \\
		reg\_t &       UUDD{T0}DD{T1} DD{T2} DDDD{T3} DD{T4} UU     UU \\
		reg\_id\_d &   UUDD{D0}DD{T1} DD{D2} DDDD{D3} DD{D4} UU     UU \\
		reg\_s\_data & UUUU    DD{RS0}DD{RS1}UUDD{RS2}DD{RS3}DD{RS4}UU \\
		reg\_t\_data & UUUU    DD{RT0}DD{RT2}UUDD{RT2}DD{RT3}DD{RT4}UU \\
		reg\_stall &   LLLL    LL     LL     HHLL     LL     LL     LL \\
	\end{tikztimingtable}
	\caption{ID Regfile接口时序}
	\label{tt:idreg}
\end{figure}

\subsection{上级FIFO接口}

\begin{signals}
	input\_address & \sigin & 32 & 指令地址 \\ \hline
	input\_instruction & \sigin & 32 & 指令数据 \\ \hline
	input\_valid & \sigin & 1 & 数据有效 \\ \hline
	input\_full & \sigout & 1 & 满反馈 \\ \hline
\end{signals}

\subsection{下级FIFO接口}

\begin{signals}
	output\_operand1 & \sigout & 32 & 操作数1 \\ \hline
	output\_operand2 & \sigout & 32 & 操作数2 \\ \hline
	output\_writereg & \sigout & 5 & 写入寄存器 \\ \hline
	output\_address & \sigout & 32 & 指令地址 \\ \hline
	output\_instruction & \sigout & 32 & 指令数据 \\ \hline
	output\_valid & \sigout & 1 & 数据有效 \\ \hline
	output\_full & \sigin & 1 & FIFO满反馈 \\ \hline
\end{signals}

\section{EX}

\subsection{时钟复位信号}

\begin{signals}
	clock & \sigin & 1 & 时钟 \\ \hline
	reset & \sigin & 1 & 同步复位,高电平有效 \\ \hline
\end{signals}

\subsection{上级FIFO接口}

\begin{signals}
	input\_operand1 & \sigin & 32 & 操作数1 \\ \hline
	input\_operand2 & \sigin & 32 & 操作数2 \\ \hline
	input\_writereg & \sigin & 5 & 写入寄存器 \\ \hline
	input\_address & \sigin & 32 & 指令地址 \\ \hline
	input\_instruction & \sigin & 32 & 指令数据 \\ \hline
	input\_valid & \sigin & 1 & 数据有效 \\ \hline
	input\_full & \sigout & 1 & 满反馈 \\ \hline
\end{signals}

\subsection{下级FIFO接口}

\begin{signals}
	output\_result & \sigout & 32 & 结果 \\ \hline
	output\_writereg & \sigout & 5 & 写入寄存器 \\ \hline
	output\_valid & \sigout & 1 & 数据有效 \\ \hline
	output\_full & \sigin & 1 & FIFO满反馈 \\ \hline
\end{signals}

\subsection{数据存储接口}

\subsubsection{信号}

\begin{signals}
	dbus\_read & \sigout & 1 & 读使能 \\ \hline
	dbus\_write & \sigout & 1 & 写使能 \\ \hline
	dbus\_address & \sigout & 32 & 地址 \\ \hline
	dbus\_byteenable & \sigout & 4 & 数据字节选择 \\ \hline
	dbus\_data & \sigout & 32 & 写数据 \\ \hline
	dbus\_stall & \sigin & 1 & 暂停 \\ \hline
\end{signals}

\subsubsection{时序}

参考图\ref{tt:exdbus}。

\begin{figure}[h]
	\centering
	\begin{tikztimingtable}
		clock &            HCCC    CC    CC    CCCC    CC    CC \\
		dbus\_read &       LLHH    HH    LL    LLLL    HH    LL \\
		dbus\_write &      LLLL    LL    HH    HHHH    LL    LL \\
		dbus\_address &    UUDD{A0}DD{A1}DD{A2}DDDD{A3}DD{A4}UU \\
		dbus\_byteenable & UUDD{B0}DD{B1}DD{B2}DDDD{B3}DD{B4}UU \\
		dbus\_data &       UUUU    UU    DD{D2}DDDD{D3}UU    UU \\
		dbus\_stall &      LLLL    LL    LL    HHLL    LL    LL \\
	\end{tikztimingtable}
	\caption{EX数据存储接口时序}
	\label{tt:exdbus}
\end{figure}

\subsection{转移输出接口}

\begin{signals}
	branch\_address & \sigout & 32 & 跳转目标 \\ \hline
	branch\_inst\_addr & \sigout & 32 & 跳转指令地址 \\ \hline
	branch\_valid & \sigout & 1 & 跳转有效 \\ \hline
\end{signals}

\section{MEM}

\subsection{时钟复位信号}

\begin{signals}
	clock & \sigin & 1 & 时钟 \\ \hline
	reset & \sigin & 1 & 同步复位,高电平有效 \\ \hline
\end{signals}

\subsection{上级FIFO接口}

\begin{signals}
	input\_result & \sigin & 32 & 结果 \\ \hline
	input\_writereg & \sigin & 5 & 写入寄存器 \\ \hline
	input\_valid & \sigin & 1 & 数据有效 \\ \hline
	input\_full & \sigout & 1 & 满反馈 \\ \hline
\end{signals}

\subsection{下级FIFO接口}

\begin{signals}
	output\_result & \sigout & 32 & 结果 \\ \hline
	output\_writereg & \sigout & 5 & 写入寄存器 \\ \hline
	output\_valid & \sigout & 1 & 数据有效 \\ \hline
	output\_full & \sigin & 1 & FIFO满反馈 \\ \hline
\end{signals}

\subsection{数据存储接口}

\subsubsection{信号}

\begin{signals}
	dbus\_data & \sigin & 32 & 读数据 \\ \hline
	dbus\_valid & \sigin & 1 & 数据有效 \\ \hline
	dbus\_enable & \sigout & 1 & 数据请求 \\ \hline
\end{signals}

\subsubsection{时序}

参考图\ref{tt:memdbus}。

\begin{figure}[h]
	\centering
	\begin{tikztimingtable}
		clock &        HCCCCC    CC    CCCC    CCCC    CC    CC \\
		dbus\_data &   UUUUDD{D0}DD{D1}UUDD{D2}DDDD{D3}DD{D4}UU \\
		dbus\_valid &  LLLLHH    HH    LLHH    HHHH    HH    LL \\
		dbus\_enable & LLHHHH    HH    HHHH    LLHH    HH    HH \\
	\end{tikztimingtable}
	\caption{MEM数据存储接口时序}
	\label{tt:memdbus}
\end{figure}

\section{WB}

\subsection{时钟复位信号}

\begin{signals}
	clock & \sigin & 1 & 时钟 \\ \hline
	reset & \sigin & 1 & 同步复位,高电平有效 \\ \hline
\end{signals}

\subsection{上级FIFO接口}

\subsubsection{信号}

\begin{signals}
	input\_result & \sigin & 32 & 结果 \\ \hline
	input\_writereg & \sigin & 5 & 写入寄存器 \\ \hline
	input\_valid & \sigin & 1 & 数据有效 \\ \hline
	input\_full & \sigout & 1 & 满反馈 \\ \hline
\end{signals}

\subsection{Regfile接口}

\begin{signals}
	reg\_wb\_d & \sigout & 5 & 写寄存器 \\ \hline
	reg\_data & \sigout & 32 & 写数据 \\ \hline
\end{signals}

\section{Regfile}

\subsection{时钟复位信号}

\begin{signals}
	clock & \sigin & 1 & 时钟 \\ \hline
	reset & \sigin & 1 & 同步复位,高电平有效 \\ \hline
\end{signals}

\subsection{存储访问信号}

\begin{signals}
	reg\_s & \sigin & 5 & rs \\ \hline
	reg\_t & \sigin & 5 & rt \\ \hline
	reg\_id\_d & \sigin & 5 & 锁寄存器 \\ \hline
	reg\_s\_data & \sigout & 32 & rs数据 \\ \hline
	reg\_t\_data & \sigout & 32 & rt数据 \\ \hline
	reg\_stall & \sigout & 1 & 暂停信号 \\ \hline
	reg\_wb\_d & \sigin & 5 & 写寄存器 \\ \hline
	reg\_d\_data & \sigin & 32 & 写数据 \\ \hline
\end{signals}

\section{指令Cache}

\subsection{时钟复位信号}

\begin{signals}
	clock & \sigin & 1 & 时钟 \\ \hline
	reset & \sigin & 1 & 同步复位,高电平有效 \\ \hline
\end{signals}

\subsection{读接口}

\begin{signals}
	ibus\_address & \sigin & 32 & 地址 \\ \hline
	ibus\_read & \sigin & 1 & 读使能 \\ \hline
	ibus\_data & \sigout & 32 & 读数据 \\ \hline
	ibus\_stall & \sigout & 1 & 暂停 \\ \hline
	ibus\_error & \sigout & 1 & 错误 \\ \hline
\end{signals}

\subsection{存储接口}

AXI-Master接口

\section{数据Cache}

\subsection{时钟复位信号}

\begin{signals}
	clock & \sigin & 1 & 时钟 \\ \hline
	reset & \sigin & 1 & 同步复位,高电平有效 \\ \hline
\end{signals}

\subsection{读写请求接口}

\begin{signals}
	dbus\_read & \sigin & 1 & 读使能 \\ \hline
	dbus\_write & \sigin & 1 & 写使能 \\ \hline
	dbus\_address & \sigin & 32 & 地址 \\ \hline
	dbus\_byteenable & \sigin & 4 & 数据字节选择 \\ \hline
	dbus\_data & \sigin & 32 & 写数据 \\ \hline
	dbus\_stall & \sigout & 1 & 暂停 \\ \hline
\end{signals}

\subsection{读回应接口}

\begin{signals}
	dbus\_data & \sigout & 32 & 读数据 \\ \hline
	dbus\_valid & \sigout & 1 & 数据有效 \\ \hline
	dbus\_enable & \sigout & 1 & 数据请求 \\ \hline
\end{signals}

\subsection{存储接口}

AXI-Master接口

\end{document}
